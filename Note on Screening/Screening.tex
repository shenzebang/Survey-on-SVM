%%%%%%%%%%%%%%%%%%%%%%%%%%%%%%%%%%%%%%%%%
% Arsclassica Article
% LaTeX Template
% Version 1.1 (10/6/14)
%
% This template has been downloaded from:
% http://www.LaTeXTemplates.com
%
% Original author:
% Lorenzo Pantieri (http://www.lorenzopantieri.net) with extensive modifications by:
% Vel (vel@latextemplates.com)
%
% License:
% CC BY-NC-SA 3.0 (http://creativecommons.org/licenses/by-nc-sa/3.0/)
%
%%%%%%%%%%%%%%%%%%%%%%%%%%%%%%%%%%%%%%%%%

%----------------------------------------------------------------------------------------
%	PACKAGES AND OTHER DOCUMENT CONFIGURATIONS
%----------------------------------------------------------------------------------------

\documentclass[
10pt, % Main document font size
a4paper, % Paper type, use 'letterpaper' for US Letter paper
oneside, % One page layout (no page indentation)
%twoside, % Two page layout (page indentation for binding and different headers)
headinclude,footinclude, % Extra spacing for the header and footer
BCOR5mm, % Binding correction
]{scrartcl}


%----------------------------------------------------------------------------------------
%	Mathematic Symbols
%----------------------------------------------------------------------------------------

\usepackage{amsmath}			% Package for AMS.
\usepackage{amsthm}				% Package for AMS-therom.
\usepackage{amssymb}			% Package for AMS-therom.

\newcommand{\AB}{\mathbf{A}}
\newcommand{\BB}{\mathbf{B}}
\newcommand{\CB}{\mathbf{C}}
\newcommand{\DB}{\mathbf{D}}
\newcommand{\EB}{\mathbf{E}}
\newcommand{\FB}{\mathbf{F}}
\newcommand{\GB}{\mathbf{G}}
\newcommand{\HB}{\mathbf{H}}
\newcommand{\IB}{\mathbf{I}}
\newcommand{\JB}{\mathbf{J}}
\newcommand{\KB}{\mathbf{K}}
\newcommand{\LB}{\mathbf{L}}
\newcommand{\MB}{\mathbf{M}}
\newcommand{\NB}{\mathbf{N}}
\newcommand{\OB}{\mathbf{O}}
\newcommand{\PB}{\mathbf{P}}
\newcommand{\QB}{\mathbf{Q}}
\newcommand{\RB}{\mathbf{R}}
\newcommand{\SB}{\mathbf{S}}
\newcommand{\TB}{\mathbf{T}}
\newcommand{\UB}{\mathbf{U}}
\newcommand{\VB}{\mathbf{V}}
\newcommand{\WB}{\mathbf{W}}
\newcommand{\XB}{\mathbf{X}}
\newcommand{\YB}{\mathbf{Y}}
\newcommand{\ZB}{\mathbf{Z}}
\newcommand{\zeroB}{\mathbf{0}}
\newcommand{\oneB}{\mathbf{1}}
\newcommand{\aB}{\mathbf{a}}
\newcommand{\bB}{\mathbf{b}}
\newcommand{\cB}{\mathbf{c}}
\newcommand{\dB}{\mathbf{d}}
\newcommand{\eB}{\mathbf{e}}
\newcommand{\fB}{\mathbf{f}}
\newcommand{\gB}{\mathbf{g}}
\newcommand{\hB}{\mathbf{h}}
\newcommand{\iB}{\mathbf{i}}
\newcommand{\jB}{\mathbf{j}}
\newcommand{\kB}{\mathbf{k}}
\newcommand{\lB}{\mathbf{l}}
\newcommand{\mB}{\mathbf{m}}
\newcommand{\nB}{\mathbf{n}}
\newcommand{\oB}{\mathbf{o}}
\newcommand{\pB}{\mathbf{p}}
\newcommand{\qB}{\mathbf{q}}
\newcommand{\rB}{\mathbf{r}}
\newcommand{\sB}{\mathbf{s}}
\newcommand{\tB}{\mathbf{t}}
\newcommand{\uB}{\mathbf{u}}
\newcommand{\vB}{\mathbf{v}}
\newcommand{\wB}{\mathbf{w}}
\newcommand{\xB}{\mathbf{x}}
\newcommand{\yB}{\mathbf{y}}
\newcommand{\zB}{\mathbf{z}}
\newcommand{\AM}{\mathcal{A}}
\newcommand{\DM}{\mathcal{D}}
\newcommand{\EM}{\mathcal{E}}
\newcommand{\FM}{\mathcal{F}}
\newcommand{\PM}{\mathcal{P}}
\newcommand{\TM}{\mathcal{T}}
\newcommand{\UM}{\mathcal{U}}
\newcommand{\VM}{\mathcal{V}}
\newcommand{\XM}{\mathcal{X}}
\newcommand{\YM}{\mathcal{Y}}
\newcommand{\LM}{\mathcal{L}}
\newcommand{\NM}{\mathcal{N}}
\newcommand{\OM}{\mathcal{O}}
\newcommand{\QM}{\mathcal{Q}}
\newcommand{\HM}{\mathcal{H}}
\newcommand{\IM}{\mathcal{I}}
\newcommand{\GM}{\mathcal{G}}
\newcommand{\SM}{\mathcal{S}}
\newcommand{\RM}{\mathcal{R}}
\newcommand{\RBB}{\mathbb{R}}
\newcommand{\xT}{\tilde{\mathbf{x}}}
\newcommand{\yT}{\tilde{\mathbf{y}}}
\newcommand{\zT}{\tilde{\mathbf{z}}}
\newcommand{\dHat}{\hat{\mathbf{d}}}
\newcommand{\DHat}{\hat{\mathbf{D}}}
\newcommand{\xHat}{\hat{\mathbf{x}}}
\newcommand{\xiB}{\mbox{\boldmath$\xi$\unboldmath}}
\newcommand{\alphaB}{\mbox{\boldmath$\alpha$\unboldmath}}
\newcommand{\betaB}{\mbox{\boldmath$\beta$\unboldmath}}
\newcommand{\etaB}{\mbox{\boldmath$\eta$\unboldmath}}
\newcommand{\epsilonB}{\mbox{\boldmath$\epsilon$\unboldmath}}
\newcommand{\phiB}{\mbox{\boldmath$\phi$\unboldmath}}
\newcommand{\piB}{\mbox{\boldmath$\pi$\unboldmath}}
\newcommand{\PhiB}{\mbox{\boldmath$\Phi$\unboldmath}}
\newcommand{\PsiB}{\mbox{\boldmath$\Psi$\unboldmath}}
\newcommand{\thetaB}{\mbox{\boldmath$\theta$\unboldmath}}
\newcommand{\ThetaB}{\mbox{\boldmath$\Theta$\unboldmath}}
\newcommand{\muB}{\mbox{\boldmath$\mu$\unboldmath}}
\newcommand{\SigmaB}{\mbox{\boldmath$\Sigma$\unboldmath}}
\newcommand{\GammaB}{\mbox{\boldmath$\Gamma$\unboldmath}}
\newcommand{\LambdaB}{\mbox{\boldmath$\Lambda$\unboldmath}}
\newcommand{\lambdaB}{\mbox{\boldmath$\lambda$\unboldmath}}
\newcommand{\omegaB}{\mbox{\boldmath$\omega$\unboldmath}}
\newcommand{\DeltaB}{\mbox{\boldmath$\Delta$\unboldmath}}
\newcommand{\varepsilonB}{\mbox{\boldmath$\varepsilon$\unboldmath}}
\newcommand{\argmax}{\mathop{\rm argmax}}
\newcommand{\argmin}{\mathop{\rm argmin}}
\newcommand{\rank}{\mathop{\rm rank}}
\newcommand{\sgn}{\mathrm{sgn}}
\newcommand{\tr}{\mathrm{tr}}
\newcommand{\rk}{\mathrm{rank}}
\newcommand{\diag}{\mathsf{diag}}
\newcommand{\dg}{\mathsf{dg}}
\newcommand{\vect}{\mathsf{vec}}
\newcommand{\etal}{{\em et al.\/}\,}
\newcommand{\ie}{{\em i.e.\/}\,}
\newcommand{\fracpartial}[2]{\frac{\partial #1}{\partial  #2}}
%----------------------------------------------------------------------------------------
%	Input Files
%----------------------------------------------------------------------------------------

\input{structure.tex} % Include the structure.tex file which specified the document structure and layout

\hyphenation{Fortran hy-phen-ation} % Specify custom hyphenation points in words with dashes where you would like hyphenation to occur, or alternatively, don't put any dashes in a word to stop hyphenation altogether

%----------------------------------------------------------------------------------------
%	TITLE AND AUTHOR(S)
%----------------------------------------------------------------------------------------

\title{\normalfont\spacedallcaps{Article Title}} % The article title

\author{\spacedlowsmallcaps{John Smith* \& James Smith\textsuperscript{1}}} % The article author(s) - author affiliations need to be specified in the AUTHOR AFFILIATIONS block

\date{} % An optional date to appear under the author(s)

%----------------------------------------------------------------------------------------

\begin{document}

%----------------------------------------------------------------------------------------
%	HEADERS
%----------------------------------------------------------------------------------------

\renewcommand{\sectionmark}[1]{\markright{\spacedlowsmallcaps{#1}}} % The header for all pages (oneside) or for even pages (twoside)
%\renewcommand{\subsectionmark}[1]{\markright{\thesubsection~#1}} % Uncomment when using the twoside option - this modifies the header on odd pages
\lehead{\mbox{\llap{\small\thepage\kern1em\color{halfgray} \vline}\color{halfgray}\hspace{0.5em}\rightmark\hfil}} % The header style

\pagestyle{scrheadings} % Enable the headers specified in this block

%----------------------------------------------------------------------------------------
%	TABLE OF CONTENTS & LISTS OF FIGURES AND TABLES
%----------------------------------------------------------------------------------------

\maketitle % Print the title/author/date block

\setcounter{tocdepth}{2} % Set the depth of the table of contents to show sections and subsections only

\tableofcontents % Print the table of contents

\listoffigures % Print the list of figures

\listoftables % Print the list of tables

%----------------------------------------------------------------------------------------
%	ABSTRACT
%----------------------------------------------------------------------------------------

\section*{Abstract} % This section will not appear in the table of contents due to the star (\section*)



%----------------------------------------------------------------------------------------
%	AUTHOR AFFILIATIONS
%----------------------------------------------------------------------------------------

{\let\thefootnote\relax\footnotetext{* \textit{Department of Biology, University of Examples, London, United Kingdom}}}

{\let\thefootnote\relax\footnotetext{\textsuperscript{1} \textit{Department of Chemistry, University of Examples, London, United Kingdom}}}

%----------------------------------------------------------------------------------------

\newpage % Start the article content on the second page, remove this if you have a longer abstract that goes onto the second page

%----------------------------------------------------------------------------------------
%	INTRODUCTION
%----------------------------------------------------------------------------------------

\section{Introduction}
My current plan is to write a survey about the screening methods used in identifying non-support vector in solving Support Vector Machine (SVM) problem.
Such methods differ from the screening strategy for lasso in that, in SVM the data points are discarded while in lasso the features are discarded.

\subsection{Margin}
Given a hyperplane $\HM = \{\xB \in \RBB^d: \wB\cdot\xB + b = 0\}$, the distance from a point $\zB \in \RBB^d$ to $\HM$ is $\DB(\zB, \HM) = \frac{\wB\cdot\zB+b}{\|\wB\|}$.
If we know whether the point is above or below the hyperplane by the sign variable $y \in \{-1, +1\}$, we define the margin to be $\MB(\zB, \HM) = y\frac{\wB\cdot\zB+b}{\|\wB\|}$

Now suppose we are given a dataset $\XM = \{y_i, \xB_i\}_{i=1}^m$, the margin of $\XM$ to $\HM$ is defined as
\begin{equation*}
	\MB(\XM, \HM) = \min_{\zB \in \XM} \MB(\zB, \HM).
\end{equation*}
Given that the dataset $\XM$ is seperable, SVM aims to seek a hyperplane $\HM^*$ that gives the largest margin of $\XM$ to $\HM^*$. Such goal can be formalized as 
\begin{equation*}
	\begin{aligned}
		\max_\HM & \quad \rho \\
		s.t. & \quad  \MB(\zB, \HM) \geq \rho, \forall \zB \in \XM
	\end{aligned}
\end{equation*}
Clearly we can scale $\wB$ and $b$ such that $\min_{\zB\in\XM} |\zB\cdot\wB+b|= 1$, thus the above formulation is equivalent to 
\begin{equation}
	\begin{aligned}
		\min & \quad \|\wB\| \\
		s.t. & \quad y_i(\xB_i\cdot\wB+b) \geq 1, i = 1, \ldots, m		
	\end{aligned}
\end{equation}
\section{SVM: non-separable case}
We are interested in solving SVM problem where there is no hyperplane that can correctly classify all the data points.
There are several formulations of SVM whose primal and dual problems, KKT conditions, and corresponding screening properties will be discussed one by one.
\subsection{Formation I}
The primal formulation of such problem is 
\begin{equation}
	\begin{aligned}
		\min_{\wB, b, \xiB} & \quad \frac{1}{2}\|\wB\|_2^2+C\sum_{i=1}^{m}\xi_i^\beta  \\
		s.t. & \quad y_i(\wB\cdot\xB_i+b)\geq 1- \xi_i \cap \xi_i \geq 0, i\in[m].
	\end{aligned}
	\tag{$P^I_\beta$}
	\label{eqn: SVM non-separable P I}
\end{equation}
where $\beta = 1$ or $2$, corresponding to $l_1$ and $l_2$ SVM variants.

It is clear that when $\beta = 1$, the $l_1$-SVM problem is equivalent to the following problem.
\begin{equation}
	\min_{\wB, b} \frac{1}{2}\|\wB\|^2+C\sum_{i=1}^{m} [1-y_i(\wB\cdot\xB_i+b)]_+
\end{equation}
Using the standard derivation, we have the following dual formulation of (\ref{eqn: SVM non-separable P I}) when $\beta = 1$
\begin{equation*}
	\begin{aligned}
		\max_{\alphaB} & \quad \|\alphaB\|_1-\frac{1}{2}\alphaB^\top\QB\alphaB \\
		s.t. & \quad 0\leq \alpha_i \leq C \cap \alphaB\cdot\yB = 0, i\in [m],
	\end{aligned}
	\tag{$D^I_1$}
	\label{eqn: SVM non-separable D I}
\end{equation*}
where $\QB_{i,j} = y_iy_j(\xB_i\cdot\xB_j)$.
Given a primal solution $\wB^*, \xiB^*, b^*$ and a dual solution $\alphaB^*$, the KKT condition states that 
\begin{enumerate}
	\item $\xi^*_i \geq 0, i = 1, \ldots, m$.
	\item $\xi^*_i \geq 1 - y_i(\wB^*\cdot\xB_i+b^*), i = 1, \ldots, m$
	\item $0 \leq \alpha^*_i \leq C, i = 1, \ldots, m$
	\item $\alpha^*_i (1 - y_i(\wB^*\cdot\xB_i+b^*) - \xi^*_i) =0, i = 1, \ldots, m$
	\item $(C - \alpha^*_i) \xi^*_i = 0, i = 1, \ldots, m$
	\item $\wB^* = \AB \alphaB^*$
\end{enumerate}
where $\AB = [\cdots y_i\xB_i \cdots]_i$.

If $1 - y_i(\wB^*\cdot\xB_i+b^*) < 0$ then $(1 - y_i(\wB^*\cdot\xB_i+b^*) - \xi^*_i) < 0$, we have $\alpha^*_i = 0$ by the complementary slackness.
Similarly, if $1 - y_i(\wB^*\cdot\xB_i+b^*) > 0$, we have $\alpha^*_i = C$.
By categorizing the $m$ training instances into three sets
\begin{itemize}
	\item $\RM := \{i\in [m] | y_i(\wB^*\cdot\xB_i+b) > 1 \}$
	\item $\EM := \{i\in [m] | y_i(\wB^*\cdot\xB_i+b) = 1 \}$
	\item $\LM := \{i\in [m] | y_i(\wB^*\cdot\xB_i+b) < 1 \}$
\end{itemize}
we summarize such property by that 
\begin{itemize}
	\item $i \in \RM \rightarrow \alpha_i = 0$
	\item $i \in \EM \rightarrow \alpha_i \in [0, C]$
	\item $i \in \LM \rightarrow \alpha_i = C$
\end{itemize}

\subsection{Formulation II}

Another way to write SVM is by 
\begin{equation}
\begin{aligned}
\min_{\wB, b, \xiB} & \quad \frac{1}{2}\|\wB\|_2^2  \\
s.t. & \quad y_i(\wB\cdot\xB_i+b)\geq 1- \xi_i \cap \xi_i \geq 0, i\in[m] \\
& \quad \sum_{i=1}^m \xi_i^\beta \leq s,
\end{aligned}
\tag{$P^{II}_\beta$}
\label{eqn: SVM non-separable P II}
\end{equation}
which is clearly equivalent to the following problem when we take $\beta  = 1$
\begin{equation}
	\begin{aligned}
		\min_{\wB, b} & \quad \frac{1}{2}\|\wB\|^2 \\
		s.t. & \quad \sum_{i=1}^m [1-y_i(\wB\cdot\xB_i+b)]_+ \leq s.
 	\end{aligned}
\end{equation}
where $[a]_+ = \max \{a, 0\}$ denotes the hinge loss.

Using the standard derivation, we have the following dual formulation of (\ref{eqn: SVM non-separable P II}) when $\beta = 1$
\begin{equation*}
	\begin{aligned}
		\max_{\alphaB, C} & \quad \|\alphaB\|_1-\frac{1}{2}\alphaB^\top\QB\alphaB - Cs\\
		s.t. & \quad 0\leq \alpha_i \leq C \cap \alphaB\cdot\yB = 0, i\in [m],
	\end{aligned}
\tag{$D^{II}_1$}
\label{eqn: SVM non-separable D II}
\end{equation*}

Given a primal solution $\wB^*, \xiB^*, b^*$ and a dual solution $\alphaB^*, C^*$, the KKT condition states that 
\begin{enumerate}
	\item $\xi^*_i \geq 0, i = 1, \ldots, m$.
	\item $\xi^*_i \geq 1 - y_i(\wB^*\cdot\xB_i+b^*), i = 1, \ldots, m$
	\item $\sum_{i=1}^m \xi^*_i \leq s$
	\item $0 \leq \alpha^*_i \leq C^*, i = 1, \ldots, m$
	\item $C^* \geq 0$
	\item $(\sum_{i=1}^m \xi^*_i - s) C^* = 0$
	\item $\alpha^*_i (1 - y_i(\wB^*\cdot\xB_i+b^*) - \xi^*_i) =0, i = 1, \ldots, m$
	\item $(C^* - \alpha^*_i) \xi^*_i = 0, i = 1, \ldots, m$
	\item $\wB^* = \AB \alphaB^*$
\end{enumerate}
where $\AB = [\cdots y_i\xB_i \cdots]_i$.

I have not yet figure out the screening properties in this formulation.
\subsection{With Kernel}

%----------------------------------------------------------------------------------------
%	Lasso
%----------------------------------------------------------------------------------------

\section{Lasso}
\subsection{Formulation I}
Primal:
\begin{equation}
	\begin{aligned}
		\min_\xB & \quad \frac{1}{2}\|\AB\xB - \bB\|^2 \\
		s.t. & \quad \|\xB\|_1 \leq t
	\end{aligned}
\end{equation}
which is equivalent to 
\begin{equation}
	\begin{aligned}
		\min_\xB & \quad \frac{1}{2}\|\zB\|^2 \\
		s.t. & \quad \|\xB\|_1 \leq t \cap \AB\xB - \bB = \zB
	\end{aligned}
	\label{eqn: Lasso constrain auxillary variable I}
\end{equation}

Dual of \ref{eqn: Lasso constrain auxillary variable I}:
\begin{equation}
	\begin{aligned}
		\max_{\thetaB, \lambda} & \quad \frac{\|\bB\|^2}{2} - \frac{\|\thetaB - \bB\|^2}{2} - \lambda t \\
		s.t. & \quad \|\AB^\top \theta\|_\infty \leq \lambda
	\end{aligned}
\end{equation}

Given a primal solution $\xB^*, \zB^*$ and dual solution $\thetaB^*, \lambda^*$ KKT condition states:
\begin{itemize}
	\item $\|\xB^*\|_1 \leq t \cap \AB\xB^* - \bB^* = \zB^*$
	\item $\lambda \geq 0$
	\item $\zB^* + \thetaB^* = 0$
	\item $\frac{\AB^\top\thetaB^*}{\lambda^*} \in \partial \|\xB^*\|_1$
	\item $\lambda (\|\xB^*\|_1 - t) = 0$
\end{itemize}
From the four$^{th}$ property, we have that if $|\AB_i \cdot \thetaB^*| < \lambda^*, \xB_i^* = 0$.

\subsection{Formulation II}
Primal:
\begin{equation}
	\min_\xB  \quad \frac{1}{2}\|\AB\xB - \bB\|^2 + \lambda \|\xB\|_1
\end{equation}
which is equivalent to 
\begin{equation}
\begin{aligned}
\min_\xB & \quad \frac{1}{2}\|\zB\|^2 + \lambda \|\xB\|_1 \\
s.t. & \quad  \AB\xB - \bB = \zB
\end{aligned}
\label{eqn: Lasso constrain auxillary variable II}
\end{equation}

Dual of \ref{eqn: Lasso constrain auxillary variable II}:
\begin{equation}
\begin{aligned}
\max_{\thetaB} & \quad \frac{\|\bB\|^2}{2} - \frac{\|\thetaB - \bB\|^2}{2} \\
s.t. & \quad \|\AB^\top \theta\|_\infty \leq \lambda
\end{aligned}
\end{equation}

Given a primal solution $\xB^*, \zB^*$ and dual solution $\thetaB^*$ KKT condition states:
\begin{itemize}
	\item $\AB\xB^* - \bB^* = \zB^*$
	\item $\frac{\AB^\top\thetaB^*}{\lambda} \in \partial \|\xB^*\|_1$
\end{itemize}
From the four$^{th}$ property, we have that if $|\AB_i \cdot \thetaB^*| < \lambda, \xB_i^* = 0$.


%----------------------------------------------------------------------------------------
%	Methods
%----------------------------------------------------------------------------------------

\section{Methods}

Support function for a ball $\BB(\cB, r) = \{\xB ~|~ \|\xB-\cB\|\leq r\}$
\begin{equation}
	\HB_{\BB(\cB, r)}(\aB) = \sup_{\xB \in \BB(\cB, r)} \aB\cdot\xB = \aB\cdot\cB + r
\end{equation}


%----------------------------------------------------------------------------------------
%	REAPER
%----------------------------------------------------------------------------------------

\section{REAPER}
Principal Component Analysis can be formalized as 
\begin{equation}
	\begin{aligned}
		\min_{\Pi} & \quad \sum_{\xB \in \XM} \|\xB - \Pi\xB\|^2 \\
		s.t. & \quad \Pi \mathrm{is~an~orthoprojector} \cap \mathrm{tr} \Pi = d
	\end{aligned}
	\label{eqn: square residual PCA}
\end{equation}
which can be readily solved by Singluar Value Decomposition (SVD) in a closed form.
However, such formulation is not robust to outliers. One alternative is to replace square residual with residual, that is 
\begin{equation}
	\begin{aligned}
		\min_{\Pi} & \quad \sum_{\xB \in \XM} \|\xB - \Pi\xB\| \\
		s.t. & \quad \Pi \mathrm{is~an~orthoprojector} \cap \mathrm{tr} \Pi = d
	\end{aligned}
	\label{eqn: residual PCA}
\end{equation}
Unfortunately, the constrain is non-convex which makes the problem difficult to solve.

Convex Relaxation is often utilized so that efficient solvers can be used. In REAPER, the constrain is relaxed to its convex hull, i.e.
\begin{equation}
	\begin{aligned}
		\min_{\PB} & \quad \sum_{\xB \in \XM} \|\xB - \PB\xB\| \\
		s.t. & \quad 0 \preccurlyeq \PB \preccurlyeq \IB  \cap \mathrm{tr} \PB = d
	\end{aligned}
	\label{eqn: REAPER I}
\end{equation}
Clearly, the solution $\PB^*$ to \ref{eqn: REAPER I} may not be feasible in the original problem \ref{eqn: residual PCA}, REAPER outputs the solution to the following problem 
\begin{equation}
	\begin{aligned}
		\min_{\Pi} & \quad \|\PB^* - \Pi\|_{S_1} \\
		s.t. & \quad \Pi \mathrm{is~an~orthoprojector} \cap \mathrm{tr} \Pi = d
	\end{aligned}
\end{equation}

\subsection{Parameters of REAPER}
Let $\MB \subset \RBB^D$ be a subspace, let $\Pi$ and $\Pi^\bot$ be the corresponding orthoprojectors to $\MB$ and $\MB^\bot$. Define the following parameters.

\emph{Permeance Statistic} $\PM(\MB)$:
\begin{equation}
	\PM(\MB) := \inf_{\uB \in \MB, \|\uB\|=1} \sum_{\xB\in\XM_{in}} |<\uB, \Pi \xB>|
\end{equation}
\emph{Total Inlier Residual} $\RM(\MB)$:
\begin{equation}
	\RM(\MB) := \sum_{\xB \in \XM_{in}} \|\Pi^\bot \xB\|.
\end{equation}
\emph{Alignment Statistic} $\AM(\MB)$:
\begin{equation}
	\AM(\MB) := \|\XB_{out}\|\cdot \|\SM(\Pi^\bot \XB_{out})\|.
\end{equation}
where $\SM(\AB)$ spherizes the columns of the matrix $\AB$.
and the \emph{Stability Statistic} $\Phi(\MB)$:
\begin{equation}
	\Phi(\MB) := \frac{\PM(\MB)}{4\sqrt{dim(\MB)}} - \AM(\MB).
\end{equation}


%----------------------------------------------------------------------------------------
%	RESULTS AND DISCUSSION
%----------------------------------------------------------------------------------------



\section{Results and Discussion}

%----------------------------------------------------------------------------------------
%	BIBLIOGRAPHY
%----------------------------------------------------------------------------------------

\renewcommand{\refname}{\spacedlowsmallcaps{References}} % For modifying the bibliography heading

\bibliographystyle{unsrt}

\bibliography{sample.bib} % The file containing the bibliography

%----------------------------------------------------------------------------------------

\end{document}